% ------------------------------------------------------------------------------
% LaTeX template created by
% Iker Algañaraz, May Juarez F., Gastón A. Lozano S., Belén N. Paz
% ------------------------------------------------------------------------------

\documentclass[a4paper,12pt]{article}

% ------------------------------------------------------------------------------
% Packages
% ------------------------------------------------------------------------------
\usepackage{anysize} % Márgenes
\usepackage[hypcap=false, font=small, justification=centering, labelfont=bf]{caption} % Pie de foto/tabla
\usepackage{multicol} % Columnas
\usepackage{amsmath} % Fórmulas matemáticas
\usepackage{amssymb} % Símbolos matemáticos
\usepackage{amsfonts} % Font matemática
\usepackage[utf8]{inputenc} % Facilitar la escritura en español
\usepackage{xcolor} % Color del texto
\usepackage{graphicx} % Figuras
\usepackage[spanish,es-tabla]{babel} % Tipografía del idioma
\usepackage{booktabs} % Separación en tablas
\usepackage{multirow} % Multirow en tablas
\usepackage{hyperref} % Refs como hyperlinks
%\usepackage{biblatex} % Bibliografía automática a partir de base bib

%\usepackage{array}
%\usepackage{verbatim} % Comentarios multilinea
%\usepackage{siunitx} % Unidades del sistema internacional
%\usepackage{fancyhdr} % Personalizar encabezado y pie de pagina
%\usepackage{longtable} % Tablas largas
%\usepackage{blindtext} % Lore ipsum
%\usepackage{soul} % Subrayar
%\usepackage{grffile}
%\usepackage{mathrsfs}

% ------------------------------------------------------------------------------
% Config
% ------------------------------------------------------------------------------
\newenvironment{Figure}
    {\par\medskip\noindent\minipage{\linewidth}}
    {\endminipage\par\medskip}

\providecommand{\keywords}[1] % Keywords
{
    \small	
    \textbf{\textit{Keywords---}} #1
}

\providecommand{\abs}[1]{\lvert#1\rvert} % Valor absoluto

\marginsize{2cm}{2cm}{1cm}{2cm} % pkg: anysize

\graphicspath{{./Fotos/}} % pkg: graphicx

\setlength\columnsep{18pt}
\setlength\parskip{4pt} \setlength\parindent{0in}

\title{Propulsion using linear induction motors\\ 
\medskip \large Universidad Nacional de Tucumán}
\author{May Juarez Ferriol}
\date{}

% ------------------------------------------------------------------------------
% Document
% ------------------------------------------------------------------------------
\begin{document}

\begin{titlepage}

    \begin{center}

        \vspace*{2cm}

        \Huge
        Estudio de fuerzas ferromagnéticas \\
        usando un motor de solenoide

        \vspace{1cm}

        \LARGE
        Juarez Ferriol May

        \vspace{1cm}

        \includegraphics[width=0.35\textwidth]{unt.jpg}

        \vspace{1cm}

        \Large
        Licenciatura en Física

        Facultad de Ciencias Exactas y Tecnología

        Universidad Nacional de Tucumán

        Tucumán, Argentina

        \vspace{1cm}

        Diciembre 2023

        \vspace{1cm}

        Supervisor: Lic. Leal Sebastián

    \end{center}

\end{titlepage}

\section*{Agradecimientos}



\section*{Resumen}

    

\section*{Introducción}

    

\section*{Marco teórico}

    \subsection*{Campo magnético de un solenoide infinito}

        Un solenoide es cualquier dispositivo físico capaz de crear un campo magnético sumamente uniforme e intenso en su interior, y muy débil en el exterior. Un ejemplo teórico es el de una bobina de hilo conductor aislado y enrollado helicoidalmente, de longitud infinita. En ese caso ideal el campo magnético sería uniforme en su interior y fuera sería nulo.

        En la práctica, una aproximación real a un solenoide es un alambre aislado, de longitud finita, enrollado en forma de hélice (bobina), por el que circula una corriente eléctrica. Cuando esto sucede, se genera un campo magnético dentro de la bobina tanto más uniforme cuanto más larga sea la bobina.

        En el caso de un solenoide infinito en el vacío, se puede obtener el módulo del campo magnético con la ecuación (\ref{eq: campoMagneticoSolenoideInfinito}), donde $\mu_0$ es la constante de permeabilidad del campo magnético, N el número de vueltas de la bobina, I la corriente, y l la longitud del solenoide.

        \begin{equation}
            \label{eq: campoMagneticoSolenoideInfinito}
            B = \frac{\mu_0 N I}{l}
        \end{equation}

    \subsection*{Campo magnético de un solenoide finito}
    
        En el caso de un solenoide finito, con x = 0 siendo el eje de la bobina, y estando en el vacío, podemos calcular el campo magnético con la ecuación (\ref{eq: campoMagneticoSolenoideFinito}).

        \begin{equation}
            \label{eq: campoMagneticoSolenoideFinito}
            B = \frac{\mu_0 N I}{2l} \left[ \frac{L/2 - x}{\sqrt{(L/2 - x)^2 + r^2}} + \frac{L/2 + x}{\sqrt{(L/2 + x)^2 + r^2}} \right]
        \end{equation}

    \subsection*{Campo magnético de un solenoide fuera del vacío}

        Las ecuaciones que hemos discutido hasta ahora son válidas para solenoides en el vacío, lo que significa que la permeabilidad del campo magnético es la del vacío, $\mu_0$.


\section*{Materiales, métodos y recolección de datos}

    

\section*{Resultados}

    

\section*{Conclusión y discusiones}

    \subsection*{Recapitulación de los objetivos}



    \subsection*{Resumen e implicación de los resultados}



    \subsection*{Futuras investigaciones}



\begin{thebibliography}{99}

    \bibitem{} D. E. Ponzetti, \emph{Proyecto experimental: Motor solenoide.} (2018)

\end{thebibliography}

\end{document}

% ------------------------------------------------------------------------------
% Common references and examples
% ------------------------------------------------------------------------------
% 
% ---------------------------
% Bibliography
% ---------------------------
% \bibitem{} Sears, Zemansky. \emph{Física universitaria}, vol. 2, 14th ed. Pearson Education, 2018.
% \bibitem{} Hecht, Zajac. \emph{Óptica}, 4th ed. Pearson Education, 2003.
% \bibitem{} Serway, Jewett. \emph{Physics for Scientists and Engineers}, vol. 2, 6th ed. Brooks Cole, 2004.
% \bibitem{} Jenkins, White. \emph{Fundamentos de óptica}, 3th ed. Aguilar S.A., 1964.
%
% ---------------------------
% Tables
% ---------------------------
% \begin{Figure}
%     \centering
%
%     \begin{tabular}{c|c}
%         \toprule
%          & \textit{...} \\
%          & \textit{[]} \\
%         \midrule
%         ... & \multirow{2}{*}{$(... \pm ...)$} \\
%         ... & \\
%         ... & \multirow{2}{*}{$(... \pm ...)$} \\
%         ... & \\ \hline
%         ... & $(... \pm ...)$ \\
%         ... & $(... \pm ...)$ \\
%         \bottomrule
%     \end{tabular}
%
%     \captionof{table}{}
%     \label{tab:}
% \end{Figure}
%
% \begin{Figure}
%     \centering
%
%     \begin{tabular}{cc}
%         \toprule
%         \textit{\textbf{... []}} & \textit{\textbf{$... []}}\\
%         \midrule
%         $... \pm ...$ & $... \pm ...$ \\
%         $... \pm ...$ & $... \pm ...$ \\
%         \bottomrule
%     \end{tabular}
%
%     \captionof{table}{}
%     \label{tab:}
% \end{Figure}
%
% ---------------------------
% Figures
% ---------------------------
% \begin{Figure}
%     \centering
%     \includegraphics[width=1\linewidth]{.jpg}
%     \captionof{figure}{}
%     \label{fig:}
% \end{Figure}
%
% ---------------------------
% Equations
% ---------------------------
% \begin{equation}
%     \label{eq:}
%     ...
% \end{equation}