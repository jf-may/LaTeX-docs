\documentclass[a4paper,12pt]{article}
% Package to change margin size
\usepackage{anysize}
\marginsize{2cm}{2cm}{1cm}{2cm}


\setlength\columnsep{18pt}

\setlength\parskip{10pt} \setlength\parindent{0in}


%TABLAS
\usepackage[table]{xcolor}
\usepackage{booktabs}
\usepackage{multirow}
\usepackage{colortbl}

\usepackage{multicol}%poder colocar columnas
\usepackage{amsmath} %formulas
\usepackage{amssymb} %simbolos
\usepackage{amsfonts} 
\usepackage{array}
\usepackage{verbatim}%escribir sin codigos y comentarios multilinea
\usepackage{xcolor}%cambiar el color del texto
\usepackage{siunitx}%unidades del sistema internacional
\usepackage{subcaption}%leyendas en objetos que tienen su leyenda
\usepackage{fancyhdr}%personalizar encabezado y pie de pagina
\usepackage{longtable}%crear tablas largas
\usepackage{hyperref}%crear e insertar links
\usepackage[utf8]{inputenc} %facilitar la escritura en español 
\usepackage{graphicx}% figuras
\usepackage[spanish,es-tabla]{babel} %tipografia del idioma
\usepackage{biblatex}%bibliografia automatica apartir de base bib
\usepackage{blindtext}%texto de relleno
\usepackage{grffile}
\usepackage{mathrsfs}
\usepackage{multirow}
\usepackage{siunitx}
\usepackage{soul}%subrayar

\renewenvironment{abstract}
 {\par\noindent\textbf{\abstractname}\ \ignorespaces \\}
 {\par\noindent\medskip}
 

\begin{document}

\begin{titlepage}
    \centering
    {\bfseries\LARGE Universidad Nacional de Tucumán \par}
    \vspace{1cm}
    {\scshape\Large Facultad de ciencias exactas y tecnologia \par}
    \vspace{3cm}
    {\scshape\Huge Informe de Laboratorio IV \par}
    \vspace{3cm}
    {\itshape\Large Formacion de imagenes por reflexión \par}
    \vfill
    {\Large Autores: \par}
    {\Large Iker Algañaraz, May Juarez F., Gastón A. Lozano S., Belén N. Paz \par}
    \vfill
\end{titlepage}
    
    
\section*{Resumen}


\section*{Marco teórico}

\subsection*{Óptica geométrica y ley de la reflexión}

La óptica geométrica se encarga del estudio de los fenómenos de la luz, mediante el uso del concepto de “Rayo”. Cuando un rayo de luz incide en la interfase entre dos medios, parte de su energía es absorbida, parte es refractada y una parte es reflejada, este último fenómeno es el que nos interesa estudiar ahora. Se encuentra experimentalmente y se ha demostrado teóricamente de distintas maneras la conocida como “Ley de la reflexión”:

\begin{equation}
    \theta_i=\theta_r
\end{equation}

Donde $\theta_i$  es el ángulo que forma el rayo incidente con la normal y $\theta_r$ es el ángulo que forma el rayo reflejado con la normal. Es decir que todo rayo incidente sale reflejado con el mismo ángulo respecto a la normal.

\subsection*{Formación de imágenes en un espejo plano}
En la figura \ref{fig:Espejo plano} se observa un objeto real colocado frente a un espejo plano. Los rayos que provenientes del objeto son y que inciden sobre el espejo salen reflejados según la ecuación (1). 

\begin{figure}[h]
    \centering  
    \includegraphics[width=0.4\textwidth]{Espejoplano-.eps}
    \caption{Diagrama espejo plano}
    \label{fig:Espejo plano}
\end{figure}

Se observa que la imagen del objeto es formada por las prolongaciones de los rayos y que se encuentra detras del espejo. Esta es una imagen "virtual", pues los rayos realmente no se intersectan ahi y solo es algo aparante para un observador. Cuando los rayos si convergen en un punto del espacio la imagen se dice "real". Se puede demostrar por trigonometria que en la figura \ref{fig:Espejo plano}:

\begin{equation}
    |p| = |q|
\end{equation}


\subsection*{Reflexión en superficies esféricas y teória paraxial}
Los espejos esféricos están formados a partir de una porción de una esfera con centro C y radio R, según la posición de la superficie reflectante estos se clasifican en cóncavos y convexos. La recta que une a C con el vértice V del espejo se llama eje óptico o eje principal del espejo como se observa en la figura \ref{fig:Espejo esferico}

\begin{figure}[h]
    \centering  
    \includegraphics[width=0.4\textwidth]{Espejoesferico-.eps}
    \caption{Diagrama espejo esférico}
    \label{fig:Espejo esferico}
\end{figure}

El estudio del comportamiento de los espejos concavos y convexos, es realizado bajo aproximaciones que constituyen la "Teoria paraxial", en la cual solo son considerados los rayos indicentes al espejo mas proximos a su eje optico.
\subsubsection*{Espejos cóncavos}
Los espejos cóncavos se caracterizan porque todos los rayos que inciden paralelos al eje óptico, salen reflejados hacia un punto común F llamado foco y la distancia desde el espejo a ese punto se llama distancia focal f.

\begin{figure}[h]
    \centering  
    \includegraphics[width=0.5\textwidth]{Espejoconcavo-.eps}
    \caption{Diagrama espejo concavo}
    \label{fig:Espejo concavo}
\end{figure}

La posición y la forma de la imagen de algún objeto formada por un espejo concavo, como la de la figura \ref{Espejo concavo} dependerá de la posición del objeto respecto a los puntos C, F y V. Mediante trigonometría y las aproximaciones de la teoría paraxial, la relación entre las distancias objeto p, distancia imagen q  y la distancia focal f se puede expresar como:

\begin{equation}
    \frac{1}{f}=\frac{1}{p}+\frac{1}{q}
\end{equation}



\subsubsection*{Espejos convexos}
Se caracterizan porque todos los rayos que inciden paralelos al eje óptico, son reflejados de manera divergente, como si salieran de un punto común al otro lado del espejo , este punto es el foco del espejo convexo. La ecuacion (3) sigue valiendo para los espejos convexos, y en la figura \ref{fig:Espejo convexo} se observa la formacion de una imagen virtual por un espejo convexo. 

\begin{figure}[h]
    \centering  
    \includegraphics[width=0.4\textwidth]{Espejoconvexo.eps}
    \caption{Diagrama espejo convexo}
    \label{fig:Espejo convexo}
\end{figure}
 
 cuando se coloca un espejo concavo o convexo justo antes de unos rayos convergentes, la imagen que se formaria actua de " objeto virtual " para el espejo, esto permite al espejo formar distintas imagenes tanto reales como virtuales. 
\section*{Desarrollo experimental}
\subsection{Espejo convexo}
Utilizamos un sistema experimental como el del diagrama X y otro como el Y, en el utilizamos la imagen real de un espejo convergente como objeto virtual del espejo convexo para generar una imagen real.Para realizar el experimento según la figura X, la colocación del espejo convexo se dificultó debido a que el mismo tapaba el camino óptico desde el objeto hasta el espejo convergente. 
Medimos las distancias imagen y objeto (Tabla X). En el caso del diagrama X, acotamos la distancia objeto e imagen considerando el valor del semi intervalo de definición del objeto y de la imagen respectivamente, la apreciación y la paralaje del espejo convexo por la manera en la que lo colocamos. Mientras que en el diagrama Y, no consideramos un error de paralaje ya que el armado del sistema experimental es más sencillo al no interferir el camino óptico del objeto real con el espejo convexo, sin embargo, en este diseño experimental la linealidad del experimento era mucho más difícil de conseguir y el objeto no era plano y existía reflexión en la ampolla de vidrio, por lo cual discernir que imagen era la correcta se dificultó mucho más.
\begin{table}
    \center
    \begin{tabular}{| c | c | c | }
        \hline
        & \bf{Distancia objeto(cm)} & \bf{Distancia imagen(cm)} \\ \hline
        & $(1,0 \pm 0,3)$ & 2 \\
        & $(2,0 \pm 0,2)$ & 4 \\
        & $(2,0 \pm 0,5)$ & 9 \\
        & 4 & 16 \\ 
        & 5 & 25 \\ \hline
    \end{tabular}
    \caption{ distancias imagen y objeto}
    \label{table: distancias}
\end{table}


Graficando el inverso de estos valores obtenemos la figura Z para los datos del diagrama X y la figura ZZ para los del diagrama Y para poder utilizar la relación 3.

% Table generated by Excel2LaTeX from sheet 'Hoja1'
\begin{table}[htbp]
  \centering
  \caption{Add caption}
    \begin{tabular}{cc|cc}
    \rowcolor[rgb]{ 1,  .753,  0} \multicolumn{2}{c|}{\textbf{Distancia objeto [mm]}} & \multicolumn{2}{c}{\cellcolor[rgb]{ .573,  .816,  .314}\textbf{Distancia Imagen [mm]}} \\
    \rowcolor[rgb]{ .851,  .851,  .851} \multicolumn{2}{c|}{20 ± 1} & \multicolumn{2}{c}{20 ± 1} \\
    \multicolumn{2}{c|}{20 ± 1} & \multicolumn{2}{c}{20 ± 1} \\
    \rowcolor[rgb]{ .851,  .851,  .851} \multicolumn{2}{c|}{20 ± 1} & \multicolumn{2}{c}{20 ± 1} \\
    \multicolumn{2}{c|}{20 ± 1} & \multicolumn{2}{c}{20 ± 1} \\
    \end{tabular}%
  \label{tab:addlabel}%
\end{table}%


\section*{Resultados}


\section*{Conclusiones}

\begin{thebibliography}{99}

\bibitem{1} Sears, Zemansky. ``\emph{Física universitaria. Vol 2"}. 14th ed. Pearson Education. (2018).



\end{thebibliography}


\end{document} 