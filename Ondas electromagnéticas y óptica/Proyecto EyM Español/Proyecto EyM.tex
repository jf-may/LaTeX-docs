% ------------------------------------------------------------------------------
% LaTeX template created by
% Iker Algañaraz, May Juarez F., Gastón A. Lozano S., Belén N. Paz
% ------------------------------------------------------------------------------

\documentclass[a4paper,12pt]{article}

% ------------------------------------------------------------------------------
% Packages
% ------------------------------------------------------------------------------
\usepackage{anysize} % Márgenes
\usepackage[hypcap=false, font=small, justification=centering, labelfont=bf]{caption} % Pie de foto/tabla
\usepackage{multicol} % Columnas
\usepackage{amsmath} % Fórmulas matemáticas
\usepackage{amssymb} % Símbolos matemáticos
\usepackage{amsfonts} % Font matemática
\usepackage[utf8]{inputenc} % Facilitar la escritura en español
\usepackage{xcolor} % Color del texto
\usepackage{graphicx} % Figuras
\usepackage[spanish,es-tabla]{babel} % Tipografía del idioma
\usepackage{booktabs} % Separación en tablas
\usepackage{multirow} % Multirow en tablas
\usepackage{hyperref} % Refs como hyperlinks
%\usepackage{biblatex} % Bibliografía automática a partir de base bib

%\usepackage{array}
%\usepackage{verbatim} % Comentarios multilinea
%\usepackage{siunitx} % Unidades del sistema internacional
%\usepackage{fancyhdr} % Personalizar encabezado y pie de pagina
%\usepackage{longtable} % Tablas largas
%\usepackage{blindtext} % Lore ipsum
%\usepackage{soul} % Subrayar
%\usepackage{grffile}
%\usepackage{mathrsfs}

% ------------------------------------------------------------------------------
% Config
% ------------------------------------------------------------------------------
\newenvironment{Figure}
    {\par\medskip\noindent\minipage{\linewidth}}
    {\endminipage\par\medskip}

\providecommand{\keywords}[1] % Keywords
{
    \small	
    \textbf{\textit{Keywords---}} #1
}

\providecommand{\abs}[1]{\lvert#1\rvert} % Valor absoluto

\marginsize{2cm}{2cm}{1cm}{2cm} % pkg: anysize

\graphicspath{{./Fotos/}} % pkg: graphicx

\setlength\columnsep{18pt}
\setlength\parskip{4pt} \setlength\parindent{0in}

\title{Propulsion using linear induction motors\\ 
\medskip \large Universidad Nacional de Tucumán}
\author{May Juarez Ferriol}
\date{}

% ------------------------------------------------------------------------------
% Document
% ------------------------------------------------------------------------------
\begin{document}

\begin{titlepage}

    \begin{center}

        \vspace*{2cm}

        \Huge
        Estudio de fuerzas ferromagnéticas \\
        usando un motor de solenoide

        \vspace{1cm}

        \LARGE
        Juarez Ferriol May

        \vspace{1cm}

        \includegraphics[width=0.35\textwidth]{unt.jpg}

        \vspace{1cm}

        \Large
        Licenciatura en Física

        Facultad de Ciencias Exactas y Tecnología

        Universidad Nacional de Tucumán

        Tucumán, Argentina

        \vspace{1cm}

        Diciembre 2023

        \vspace{1cm}

        Supervisor: Lic. Leal Sebastián

    \end{center}

\end{titlepage}

\section*{Agradecimientos}



\section*{Resumen}

    

\section*{Introducción}

    

\section*{Marco teórico}

    \subsection*{Campo magnético de un solenoide infinito en el vacío}

        Un solenoide es cualquier dispositivo físico capaz de crear un campo magnético sumamente uniforme e intenso en su interior, y muy débil en el exterior. Un ejemplo teórico es el de una bobina de hilo conductor aislado y enrollado helicoidalmente, de longitud infinita. En ese caso ideal el campo magnético sería uniforme en su interior y fuera sería nulo.

        En la práctica, una aproximación real a un solenoide es un alambre aislado, de longitud finita, enrollado en forma de hélice (bobina), por el que circula una corriente eléctrica. Cuando esto sucede, se genera un campo magnético dentro de la bobina tanto más uniforme cuanto más larga sea la bobina.

        En el caso de un solenoide infinito en el vacío, se puede obtener el módulo del campo magnético en el eje axial con la ecuación (\ref{eq: campoMagneticoSolenoideInfinito}), donde $\mu_0$ es la constante de permeabilidad magnética en el vacío, N el número de espiras del solenoide, I la corriente que circula, y L la longitud total del solenoide.

        \begin{equation}
            \label{eq: campoMagneticoSolenoideInfinito}
            B_x = \frac{\mu_0 N I}{L}
        \end{equation}

    \subsection*{Campo magnético de un solenoide finito en el vacío}
    
        En el caso de un solenoide finito, con x = 0 siendo el centro del solenoide, y estando en el vacío, podemos calcular el campo magnético en el eje axial con la ecuación (\ref{eq: campoMagneticoSolenoideFinito}).

        \begin{equation}
            \label{eq: campoMagneticoSolenoideFinito}
            B_x = \frac{\mu_0 N I}{2L} \left[ \frac{L/2 - x}{\sqrt{(L/2 - x)^2 + r^2}} + \frac{L/2 + x}{\sqrt{(L/2 + x)^2 + r^2}} \right]
        \end{equation}

    \subsection*{Campo magnético de un solenoide con núcleo ferromagnético}

        Las ecuaciones que hemos discutido hasta ahora son válidas para solenoides en el vacío, lo que significa que la permeabilidad del campo magnético es la misma que la del vacío, $\mu_0$.

        Si el solenoide está inmerso en un material con permeabilidad relativa $\mu_r$, entonces el campo se incrementa proporcionalmente a esa cantidad, que podemos ver modificando la ecuación (\ref{eq: campoMagneticoSolenoideInfinito}), obteniendo la ecuación (\ref{eq: campoMagneticoSolenoideInfinitoRelativo}).

        \begin{equation}
            \label{eq: campoMagneticoSolenoideInfinitoRelativo}
            B_x = \frac{\mu_r\mu_0 N I}{L}
        \end{equation}

        En la mayoría de los solenoides, éste no se encuentra sumergido en un material de mayor permeabilidad, si no que parte del espacio alrededor del solenoide tiene el material de mayor permeabilidad, y parte del espacio alrededor es simplemente aire, que se comporta muy parecido al vacío. En ese caso, el efecto completo del material de mayor permeabilidad no se ve, si no que va a existir una permabilidad efectiva, o aparente, $\mu_{ef}$ de manera que $1 \leq \mu_{ef} \leq \mu_r$.

        La inclusión de un nucleo ferromagnético aumenta la permeabilidad efectiva del campo magnético, como se puede ver para un solenoide infinito en la ecuación (\ref{eq: campoMagneticoSolenoideInfinitoEfectivo}).

        \begin{equation}
            \label{eq: campoMagneticoSolenoideInfinitoEfectivo}
            B_x = \frac{\mu_{ef}\mu_0 N I}{L}
        \end{equation}

    \subsection*{Campo magnético de un solenoide finito con núcleo ferromagnético}

        Si juntamos lo discutido anteriormente sobre el solenoide finito y el núcleo ferromagnético, llegamos a la ecuación (\ref{eq: campoMagneticoSolenoideFinitoEfectivo}).

        \begin{equation}
            \label{eq: campoMagneticoSolenoideFinitoEfectivo}
            B_x = \frac{\mu_{ef}\mu_0 N I}{2L} \left[ \frac{L/2 - x}{\sqrt{(L/2 - x)^2 + r^2}} + \frac{L/2 + x}{\sqrt{(L/2 + x)^2 + r^2}} \right]
        \end{equation}

        En los extremos del solenoide (obtención de la ecuación en el apéndice A), el campo magnético es igual a la ecuación (\ref{eq: campoMagneticoSolenoideFinitoEfectivoExtremos}).

        \begin{equation}
            \label{eq: campoMagneticoSolenoideFinitoEfectivoExtremos}
            B_x = \frac{\mu_{ef}\mu_0 N I}{2\sqrt{L^2 + r^2}}
        \end{equation}

        Como se esperaría de un sistema en donde la dirección negativa y positiva es arbitraria y no modifica nada, los campos magnéticos son simétricos respecto al punto central del solenoide.


\section*{Materiales, métodos y recolección de datos}

    

\section*{Resultados}

    

\section*{Conclusión y discusiones}

    \subsection*{Recapitulación de los objetivos}



    \subsection*{Resumen e implicación de los resultados}



    \subsection*{Futuras investigaciones}



\begin{thebibliography}{99}

    \bibitem{} D. E. Ponzetti, \emph{Proyecto experimental: Motor solenoide.} (2018)

\end{thebibliography}

\section*{Apéndice A - Campo magnético en los extremos del \\solenoide}

    \begin{equation*}
        B_x = \frac{\mu_{ef}\mu_0 N I}{2L} \left[ \frac{L/2 - x}{\sqrt{(L/2 - x)^2 + r^2}} + \frac{L/2 + x}{\sqrt{(L/2 + x)^2 + r^2}} \right]
    \end{equation*}

    En el extremo izquierdo ($x=-L/2$)

    \begin{equation*}
        B_x = \frac{\mu_{ef}\mu_0 N I}{2L} \left[ \frac{L/2 - (-L/2)}{\sqrt{(L/2 - (-L/2))^2 + r^2}} + \frac{L/2 + (-L/2)}{\sqrt{(L/2 + (-L/2))^2 + r^2}} \right]
    \end{equation*}

    \begin{equation*}
        B_x = \frac{\mu_{ef}\mu_0 N I}{2L} \left[ \frac{L/2 + L/2}{\sqrt{(L/2 + L/2)^2 + r^2}} + \frac{L/2 - L/2}{\sqrt{(L/2 - L/2)^2 + r^2}} \right]
    \end{equation*}

    \begin{equation*}
        B_x = \frac{\mu_{ef}\mu_0 N I}{2L} \left[ \frac{L}{\sqrt{L^2 + r^2}} \right]
    \end{equation*}

    \begin{equation*}
        B_x = \frac{\mu_{ef}\mu_0 N I}{2\sqrt{L^2 + r^2}}
    \end{equation*}

De manera análoga en el extremo derecho ($x=L/2$)

    \begin{equation*}
        B_x = \frac{\mu_{ef}\mu_0 N I}{2L} \left[ \frac{L/2 - (L/2)}{\sqrt{(L/2 - (L/2))^2 + r^2}} + \frac{L/2 + (L/2)}{\sqrt{(L/2 + (L/2))^2 + r^2}} \right]
    \end{equation*}

    \begin{equation*}
        B_x = \frac{\mu_{ef}\mu_0 N I}{2L} \left[ \frac{L/2 - L/2}{\sqrt{(L/2 - L/2)^2 + r^2}} + \frac{L/2 + L/2}{\sqrt{(L/2 + L/2)^2 + r^2}} \right]
    \end{equation*}

    \begin{equation*}
        B_x = \frac{\mu_{ef}\mu_0 N I}{2L} \left[ \frac{L}{\sqrt{L^2 + r^2}} \right]
    \end{equation*}
    
    \begin{equation*}
        B_x = \frac{\mu_{ef}\mu_0 N I}{2\sqrt{L^2 + r^2}}
    \end{equation*}

\end{document}

% ------------------------------------------------------------------------------
% Common references and examples
% ------------------------------------------------------------------------------
% 
% ---------------------------
% Bibliography
% ---------------------------
% \bibitem{} Sears, Zemansky. \emph{Física universitaria}, vol. 2, 14th ed. Pearson Education, 2018.
% \bibitem{} Hecht, Zajac. \emph{Óptica}, 4th ed. Pearson Education, 2003.
% \bibitem{} Serway, Jewett. \emph{Physics for Scientists and Engineers}, vol. 2, 6th ed. Brooks Cole, 2004.
% \bibitem{} Jenkins, White. \emph{Fundamentos de óptica}, 3th ed. Aguilar S.A., 1964.
%
% ---------------------------
% Tables
% ---------------------------
% \begin{Figure}
%     \centering
%
%     \begin{tabular}{c|c}
%         \toprule
%          & \textit{...} \\
%          & \textit{[]} \\
%         \midrule
%         ... & \multirow{2}{*}{$(... \pm ...)$} \\
%         ... & \\
%         ... & \multirow{2}{*}{$(... \pm ...)$} \\
%         ... & \\ \hline
%         ... & $(... \pm ...)$ \\
%         ... & $(... \pm ...)$ \\
%         \bottomrule
%     \end{tabular}
%
%     \captionof{table}{}
%     \label{tab:}
% \end{Figure}
%
% \begin{Figure}
%     \centering
%
%     \begin{tabular}{cc}
%         \toprule
%         \textit{\textbf{... []}} & \textit{\textbf{$... []}}\\
%         \midrule
%         $... \pm ...$ & $... \pm ...$ \\
%         $... \pm ...$ & $... \pm ...$ \\
%         \bottomrule
%     \end{tabular}
%
%     \captionof{table}{}
%     \label{tab:}
% \end{Figure}
%
% ---------------------------
% Figures
% ---------------------------
% \begin{Figure}
%     \centering
%     \includegraphics[width=1\linewidth]{.jpg}
%     \captionof{figure}{}
%     \label{fig:}
% \end{Figure}
%
% ---------------------------
% Equations
% ---------------------------
% \begin{equation}
%     \label{eq:}
%     ...
% \end{equation}