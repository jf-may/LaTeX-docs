% ------------------------------------------------------------------------------
% LaTeX template created by
% May Juarez Ferriol
% ------------------------------------------------------------------------------

\documentclass[a4paper,twocolumn,12pt]{article}

% ------------------------------------------------------------------------------
% Packages and configs
% ------------------------------------------------------------------------------
\usepackage{anysize} % Margins
\marginsize{2cm}{2cm}{1cm}{2cm}

%\usepackage{multicol} % Columns
%\setlength\columnsep{18pt} \setlength\parskip{4pt} \setlength\parindent{0in}

\usepackage{graphicx} % Images
\graphicspath{{./images/}}

\usepackage[hypcap=false, font=small, justification=centering, labelfont=bf]{caption} % Footnote

\usepackage{amsmath} % Math equations
\usepackage{amssymb} % Math symbols

\usepackage{biblatex} % Bibliography

\usepackage[spanish,es-tabla]{babel} % Español

\usepackage{hyperref} % Hyperlinks

\usepackage{array} % Tables

\usepackage{fancyhdr} % Heading and subheading

\usepackage{csquotes} % Quotes

%\usepackage{multirow} % Multirow in tables

% ------------------------------------------------------------------------------
% Config
% ------------------------------------------------------------------------------
\newenvironment{Figure}
  {\par\medskip\noindent\minipage{\linewidth}}
  {\endminipage\par\medskip}

\providecommand{\abs}[1]{\lvert#1\rvert} % Valor absoluto

\title{ ... \\ 
\medskip \large Universidad Nacional de Tucumán}
\author{May Juarez Ferriol}
\date{\today}

% ------------------------------------------------------------------------------
% Document
% ------------------------------------------------------------------------------
\begin{document}

\maketitle

\begin{abstract}



\end{abstract} 

\section*{Introducción}

    

\section*{Marco teórico}

    

\section*{Método experimental}

    

\section*{Datos y resultados}

    

\section*{Conclusiones}

    

% \begin{thebibliography}{99}



% \end{thebibliography}

\end{document}

% ------------------------------------------------------------------------------
% Common references and examples
% ------------------------------------------------------------------------------
% 
% ---------------------------
% Bibliography
% ---------------------------
% \bibitem{} Sears, Zemansky. \emph{Física universitaria}, vol. 2, 14th ed. Pearson Education, 2018.
% \bibitem{} Hecht, Zajac. \emph{Óptica}, 4th ed. Pearson Education, 2003.
% \bibitem{} Serway, Jewett. \emph{Physics for Scientists and Engineers}, vol. 2, 6th ed. Brooks Cole, 2004.
% \bibitem{} Jenkins, White. \emph{Fundamentos de óptica}, 3th ed. Aguilar S.A., 1964.
%
% ---------------------------
% Tables
% ---------------------------
% \begin{Figure}
%     \centering
%
%     \begin{tabular}{c|c}
%         \toprule
%          & \textit{...} \\
%          & \textit{[]} \\
%         \midrule
%         ... & \multirow{2}{*}{$(... \pm ...)$} \\
%         ... & \\
%         ... & \multirow{2}{*}{$(... \pm ...)$} \\
%         ... & \\ \hline
%         ... & $(... \pm ...)$ \\
%         ... & $(... \pm ...)$ \\
%         \bottomrule
%     \end{tabular}
%
%     \captionof{table}{}
%     \label{tab:}
% \end{Figure}
%
% \begin{Figure}
%     \centering
%
%     \begin{tabular}{cc}
%         \toprule
%         \textit{\textbf{... []}} & \textit{\textbf{$... []}}\\
%         \midrule
%         $... \pm ...$ & $... \pm ...$ \\
%         $... \pm ...$ & $... \pm ...$ \\
%         \bottomrule
%     \end{tabular}
%
%     \captionof{table}{}
%     \label{tab:}
% \end{Figure}
%
% ---------------------------
% Figures
% ---------------------------
% \begin{Figure}
%     \centering
%     \includegraphics[width=1\linewidth]{.jpg}
%     \captionof{figure}{}
%     \label{fig:}
% \end{Figure}
%
% ---------------------------
% Equations
% ---------------------------
% \begin{equation}
%     \label{eq:}
%     ...
% \end{equation}